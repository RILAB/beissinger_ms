\documentclass[]{article}

\begin{document}

Dear Editor, 
Please find enclosed a manuscript entitled, ``Recent demography drives changes in linked selection across the maize genome''. 
Many organisms, from humans to domesticated crops to models such as \texit{Arabidopsis} and \texti{Drosophila}, have undergone significant demographic change in their recent past.  
Such recent changes have often been ignored because their impact on total levels of genetic diversity and long-term effective population size ($N_e$) is usually minimal.

%still working
But a single number such as N e fails to encapsulate all of a population’s demographic history. 
We used resequencing data from domesticated maize and its wild ancestor teosinte to investigate the impacts of dynamic population size changes on patterns of linked selection. By comparing diversity in allelic classes of different ages, we show that the impact of purifying selection on linked diversity in maize and teosinte dramatically shifts as a result of the rapid expansion of maize post­domestication. This observation has broad implications across the tree of life: Our work shows that a detailed understanding of the role of linked selection and the fate of new mutations depends critically on these recent demographic shifts and is not captured well using only estimates of long­term N e. 
We hope that our observations regarding the impact of population size changes on patterns of selection and genome­wide diversity appeal to the broad readership of PLoS Genetics, including researchers who work on any species across the genetics and genomics spectrum. 
Thank you,
Tim Beissinger & Jeffrey Ross­-Ibarra 

\end{document}